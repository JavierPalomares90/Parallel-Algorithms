%
% This is the LaTeX template file for lecture notes for EE 382C/EE 361C.
%
% To familiarize yourself with this template, the body contains
% some examples of its use.  Look them over.  Then you can
% run LaTeX on this file.  After you have LaTeXed this file then
% you can look over the result either by printing it out with
% dvips or using xdvi.
%
% This template is based on the template for Prof. Sinclair's CS 270.

\documentclass[twoside]{article}
\usepackage{graphics}
\setlength{\oddsidemargin}{0.25 in}
\setlength{\evensidemargin}{-0.25 in}
\setlength{\topmargin}{-0.6 in}
\setlength{\textwidth}{6.5 in}
\setlength{\textheight}{8.5 in}
\setlength{\headsep}{0.75 in}
\setlength{\parindent}{0 in}
\setlength{\parskip}{0.1 in}

%
% The following commands set up the lecnum (lecture number)
% counter and make various numbering schemes work relative
% to the lecture number.
%
\newcounter{lecnum}
\renewcommand{\thepage}{\thelecnum-\arabic{page}}
\renewcommand{\thesection}{\thelecnum.\arabic{section}}
\renewcommand{\theequation}{\thelecnum.\arabic{equation}}
\renewcommand{\thefigure}{\thelecnum.\arabic{figure}}
\renewcommand{\thetable}{\thelecnum.\arabic{table}}

%
% The following macro is used to generate the header.
%
\newcommand{\lecture}[4]{
   \pagestyle{myheadings}
   \thispagestyle{plain}
   \newpage
   \setcounter{lecnum}{#1}
   \setcounter{page}{1}
   \noindent
   \begin{center}
   \framebox{
      \vbox{\vspace{2mm}
    \hbox to 6.28in { {\bf EE 382V: Parallel Algorithms
                        \hfill Summer 2017} }
       \vspace{4mm}
       \hbox to 6.28in { {\Large \hfill Lecture #1: #2  \hfill} }
       \vspace{2mm}
       \hbox to 6.28in { {\it Lecturer: #3 \hfill Scribe: #4} }
      \vspace{2mm}}
   }
   \end{center}
   \markboth{Lecture #1: #2}{Lecture #1: #2}
   %{\bf Disclaimer}: {\it These notes have not been subjected to the
   %usual scrutiny reserved for formal publications.  They may be distributed
   %outside this class only with the permission of the Instructor.}
   \vspace*{4mm}
}

%
% Convention for citations is authors' initials followed by the year.
% For example, to cite a paper by Leighton and Maggs you would type
% \cite{LM89}, and to cite a paper by Strassen you would type \cite{S69}.
% (To avoid bibliography problems, for now we redefine the \cite command.)
% Also commands that create a suitable format for the reference list.
\renewcommand{\cite}[1]{[#1]}
\def\beginrefs{\begin{list}%
        {[\arabic{equation}]}{\usecounter{equation}
         \setlength{\leftmargin}{2.0truecm}\setlength{\labelsep}{0.4truecm}%
         \setlength{\labelwidth}{1.6truecm}}}
\def\endrefs{\end{list}}
\def\bibentry#1{\item[\hbox{[#1]}]}

%Use this command for a figure; it puts a figure in wherever you want it.
%usage: \fig{NUMBER}{SPACE-IN-INCHES}{CAPTION}
\newcommand{\fig}[3]{
			\vspace{#2}
			\begin{center}
			Figure \thelecnum.#1:~#3
			\end{center}
	}
% Use these for theorems, lemmas, proofs, etc.
\newtheorem{theorem}{Theorem}[lecnum]
\newtheorem{lemma}[theorem]{Lemma}
\newtheorem{proposition}[theorem]{Proposition}
\newtheorem{claim}[theorem]{Claim}
\newtheorem{corollary}[theorem]{Corollary}
\newtheorem{definition}[theorem]{Definition}
\newenvironment{proof}{{\bf Proof:}}{\hfill\rule{2mm}{2mm}}

% **** IF YOU WANT TO DEFINE ADDITIONAL MACROS FOR YOURSELF, PUT THEM HERE:
%\newcommand*{\expct}{\mathsf{E}}
%\newcommand*{\prob1}{\mathsf{P}}
\usepackage{amsmath,amssymb}
\DeclareMathOperator{\E}{\mathbb{E}}
\DeclareMathOperator{\Prob}{\mathbb{P}}


\begin{document}
%FILL IN THE RIGHT INFO.
%\lecture{**LECTURE-NUMBER**}{**DATE**}{**LECTURER**}{**SCRIBE**}
\lecture{16}{July 1}{Vijay Garg}{John Martinez}
%\footnotetext{These notes are partially based on those of Nigel Mansell.}

% **** YOUR NOTES GO HERE:

% Some general latex examples and examples making use of the
% macros follow.  
%**** IN GENERAL, BE BRIEF. LONG\, SCRIBE NOTES, NO MATTER HOW WELL WRITTEN,
%**** ARE NEVER READ BY ANYBODY.
\section{Introduction}
This section will involve the notion of estimating probability (bound calculation) by using the following methods: {\tt Markov, Chebyshev} and {\tt Chernoff}.
The following notes are based on {\em Probability and Computation} by Mitzenmacher and Upfal, pages 44-49 \& 64. 
%You can run {\tt pdflatex} on that file to generate {\em scribe.pdf}. The remaining document shows usage of some of the commands in Latex.

\section{Flipping coin example and other generalities}
Given a coin $x_{i}$, where $0$ is tails and heads is $1$, the probability $\Prob$ that once flipped the coin will be heads can be denoted:
\[\Prob[x_{i}=1] = \frac{1}{2},\;\;\;\; x_{i}=\begin{cases} 0 \\ 1 \end{cases}\]
The expected value $\E$ of such coin $x_{i}$, also called the average, is:
\[\E[x_{i}] = \frac{1}{2}\]

If you flip it $n$ times then we obtain a random variable $X$:
\[ X = \sum_{i=1}^{n} x_{i}\]

\subsection{Linearity of Expectation aka Golden Rule}
The expected value of a sum of random variables is equal to the sum of the individual expectation. Hence, given random variables $x_{i}$ and not assuming any relationship among them, then: \\
\[\forall x_{i}\:|\:X = x_{1} + x_{2} + \ldots + x_{n} \Longleftrightarrow \E[X] = \E[x_{1}] + \E[x_{2}] + \ldots + \E[x_{n}]\]
Using the coin example, and since $\sum\E=\E\sum$, then:
\[\E[X] = \sum_{i=1}^{n} \E[x_{i}] = \frac{1}{2}\sum_{i=1}^{n} 1 = \frac{n}{2} \]
% http://www.cse.iitd.ac.in/~mohanty/col106/Resources/linearity_expectation.pdf

%The naive and obvious solution to All Pairs Shortest Path(APSP) problem is to run a Single Source Shortest Path algorithm from each starting vertex $v$.  If the graph has arbitrary edge weights, it takes the Bellman-Ford algorithm $O(|E||V|^2)$ time to solve APSP.  But there are better approaches.

\section{Markov's Inequality}
%https://en.wikipedia.org/wiki/Markov%27s_inequality
This inequality relate probabilities to expectations by giving an upper bound for the probability that a non-negative function of a random variable is greater than or equal to some positive constant. Thus, given a random variable X 

%Label the vertices $1,2,\ldots,n$. Define $d^{(k)}(i,j)$ to be the length of a shortest path from $i$ to $j$, using intermediate vertices from \{$1,2,\ldots,k$\} only. Obviously, $d^{(n)}(i,j)$ is the full problem.  

\section{Chebyshev's Inequality}
%https://en.wikipedia.org/wiki/Chebyshev%27s_inequality

%Our goal is to achieve running time $O(M(n)\log n)$ for APSP where$M(n)$ is the time for $n\times n$ matrix multiplication.  Let's see if we can achieve this for a simpler but related problem, namely {\it Transitive Closure\/}:

\section{Chernoff Bound}
%https://en.wikipedia.org/wiki/Chernoff_bound

\ldots

\section*{References}
\beginrefs
\bibentry{AGM97}{\sc N.~Alon}, {\sc Z.~Galil} and {\sc O.~Margalit},
On the Exponent of the All Pairs Shortest Path Problem,
{\it Journal of Computer and System Sciences\/}~{\bf 54} (1997),
pp.~255--262.

\bibentry{F76}{\sc M. L. ~Fredman}, New Bounds on the Complexity of the 
Shortest Path Problem, {\it SIAM Journal on Computing\/}~{\bf 5} (1976), 
pp.~83-89.
\endrefs


\end{document}




